%\documentclass[mathserif]{beamer}
\documentclass[handout]{beamer}
%\usetheme{Goettingen}
\usetheme{Warsaw}
%\usetheme{Singapore}
%\usetheme{Frankfurt}
%\usetheme{Copenhagen}
%\usetheme{Szeged}
%\usetheme{Montpellier}
%\usetheme{CambridgeUS}
%\usecolortheme{}
%\setbeamercovered{transparent}
\usepackage[english, activeacute]{babel}
\usepackage[utf8]{inputenc}
\usepackage{amsmath, amssymb}
\usepackage{dsfont}
\usepackage{graphics}
\usepackage{cases}
\usepackage{graphicx}
\usepackage{pgf}
\usepackage{epsfig}
\usepackage{amssymb}
\usepackage{multirow}	
\usepackage{amstext}
\usepackage[ruled,vlined,lined]{algorithm2e}
\usepackage{amsmath}
\usepackage{epic}
\usepackage{epsfig}
\usepackage{fontenc}
\usepackage{framed,color}
\usepackage{palatino, url, multicol}
\usepackage{listings}
%\algsetup{indent=2em}


\vspace{-0.5cm}
\title{Generalized and Multilevel Linear Models}
\vspace{-0.5cm}
\author[Felipe Bravo Márquez]{\footnotesize
%\author{\footnotesize  
 \textcolor[rgb]{0.00,0.00,1.00}{Felipe José Bravo Márquez}} 
\date{ \today }




\begin{document}
\begin{frame}
\titlepage


\end{frame}


%%%%%%%%%%%%%%%%%%%%%%%%%%%


\begin{frame}{Generalized and Multilevel Linear Models}
\scriptsize{
\begin{itemize}
\item In this class we will learn two powerful extensions to the linear model, which we have discussed extensively throughout this course.

\item The first extensions is the \textbf{Generalized Linear Model} (GLM) which allows the use of distributions other than Gaussian in the outcome variable.

\item GLMs can be particularly useful when our outcome variable  is binary or bounded to positive values.

\item \textbf{Multilevel models} (also known as hierarchical or mixed effects models), on the other hand, are useful when there are predictors at different level of variation.

\item For example, when studying student performance, we may have information at different levels:  individual students  (e.g., family background), class-level information (e.g., teacher), and school-level information (e.g., neighborhood) \cite{gelman2013bayesian}.


\item Multilevel models extend linear regression to include categorical input variable representing these levels, while allowing intercepts and possibly slopes to vary by level \cite{gelman2006data}.



\end{itemize}



}

\end{frame}


\begin{frame}{Generalized Linear Models}
\scriptsize{
\begin{itemize}
\item The linear regression models of previous classes worked by first assuming a Gaussian distribution over outcomes.
\item Then, we replaced the parameter that defines the mean of that distribution, $\mu$, with a linear model.
\item This resulted in likelihood definitions of the sort:

 \vspace{0.3cm}
 \begin{table}
 \centering
 \begin{tabular}{lr}
$y_i \sim N(\mu_i,\sigma)$ & [likelihood] \\
$\mu_i = \beta_0 + \beta_1 x_i$ & [linear model] \\
\end{tabular}
\end{table}
 \vspace{0.3cm}

\item When the outcome variable is either discrete or bounded, a Gaussian likelihood is not the most powerful choice.

\item Consider for example a count outcome, such as the number of blue marbles pulled from a bag.
\item Such a variable is constrained to be zero or a positive integer.

 
\end{itemize}



} 

\end{frame}

\begin{frame}{Generalized Linear Models}
\scriptsize{
\begin{itemize}


\item The problem of using a Gaussian model with such a variable is that the model wouldn't know that counts can't be negative.

\item So it would happily predict negative values, whenever the mean count is close to zero \cite{mcelreath2020statistical}.

\item In linear regression we basically replace the parameter describing the shape of the Gaussian likelihood $\mu$ with a linear model.

\item The the essence of a Generalized Linear Model (GLM) is to generalize this strategy to probability distributions other than the Gaussian.

\item And it results in models that look like this:

 \vspace{0.3cm}
 \begin{table}
 \centering
 \begin{tabular}{lr}
$y_i \sim \text{Binomial}(n,p_i)$ & [likelihood] \\
$f(p_i) = \beta_0 + \beta_1 x_i$ & [generalized linear model] \\
\end{tabular}
\end{table}
 \vspace{0.3cm}


\end{itemize}



}

\end{frame}


\begin{frame}{Generalized Linear Models}
\scriptsize{
\begin{itemize}


\item  The first change we can note is that likelihood is binomial instead of Gaussian.
\item For a count outcome $y$ for which each observation arises from $n$ trials and with constant expected value $n*p$, the binomial distribution is the de facto choice.

\item The function $f$ represents a \textbf{link} function, to be determined separately from the choice of distribution.

\item Generalized linear models need a link function, because rarely is there a ``$\mu$'', a parameter describing the average outcome.
\item Parameters are also rarely unbounded in both directions, like $\mu$.
\item For example, the shape of the binomial distribution is determined, like the Gaussian, by two parameters.
\end{itemize}



}

\end{frame}


\begin{frame}{Generalized Linear Models}
\scriptsize{
\begin{itemize}

\item But unlike the Gaussian, neither of these parameters is the mean.
\item Instead, the mean outcome is $n*p$, which is a function of both parameters.
\item Since $n$ is usually known (but not always), it is most common to attach a linear model to the unknown part, $p$.
\item But $p$ is a probability, so $p_i$ must lie between zero and one.
\item But there's nothing to stop the linear model $\beta_0 + \beta_1 x_i$ from falling below zero or exceeding one.
\item The \textbf{link} function $f$ provides a solution to this common problem.

\item The link function that is commonly used when working with binomial GLMs is the logit function.

\item It maps a parameter that is defined as a probability $p$ (i.e., $0\leq p\leq 1$), onto a linear model that can take on any real value.

 \vspace{0.3cm}
 \begin{table}
 \centering
 \begin{tabular}{l}
logit$(p_i) = \beta_0 + \beta_1 x_i$ \\ \\
logit$(p_i) = \log\left(\frac{p_i}{1-p_i}\right)$
\end{tabular}
\end{table}
 \vspace{0.3cm}


\end{itemize}



}

\end{frame}


\begin{frame}{Generalized Linear Models}
\scriptsize{
\begin{itemize}

\item If we solve the logit equation for $p_i$ we get:

\begin{equation}
 p_i = \frac{\exp (\beta_0 + \beta_1 x_i)}{1+\exp (\beta_0 + \beta_1 x_i)}
\end{equation}

which is known as the \textbf{sigmoid} function.

\item It is also called the \textbf{logistic} or the \textbf{inverse-logit} function.

\begin{figure}[h!]
	\centering
	\includegraphics[scale=0.4]{pics/sigmoid.png}
\end{figure}

The logit link transforms a linear model (left) into a probability
(right).

\end{itemize}


}

\end{frame}


\begin{frame}{Binomial Regression}
\scriptsize{
\begin{itemize}

\item There are two common flavors of GLM that use binomial likelihood functions:

\end{itemize}


}

\end{frame}


\begin{frame}{Other types of GLMs}
\scriptsize{
\begin{itemize}

\item There are two common flavors of GLM that use binomial likelihood functions:

\end{itemize}


}

\end{frame}


\begin{frame}{Conclusions}
\scriptsize{

\begin{itemize}
\item Blabla
\end{itemize}


} 
\end{frame}


%%%%%%%%%%%%%%%%%%%%%%%%%%%
\begin{frame}[allowframebreaks]\scriptsize
\frametitle{References}
\bibliography{bio}
\bibliographystyle{apalike}
%\bibliographystyle{flexbib}
\end{frame}  









%%%%%%%%%%%%%%%%%%%%%%%%%%%

\end{document}
