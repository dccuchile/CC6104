%\documentclass[mathserif]{beamer}
\documentclass[handout]{beamer}
%\usetheme{Goettingen}
\usetheme{Warsaw}
%\usetheme{Singapore}
%\usetheme{Frankfurt}
%\usetheme{Copenhagen}
%\usetheme{Szeged}
%\usetheme{Montpellier}
%\usetheme{CambridgeUS}
%\usecolortheme{}
%\setbeamercovered{transparent}
\usepackage[english, activeacute]{babel}
\usepackage[utf8]{inputenc}
\usepackage{amsmath, amssymb}
\usepackage{dsfont}
\usepackage{graphics}
\usepackage{cases}
\usepackage{graphicx}
\usepackage{pgf}
\usepackage{epsfig}
\usepackage{amssymb}
\usepackage{multirow}	
\usepackage{amstext}
\usepackage[ruled,vlined,lined]{algorithm2e}
\usepackage{amsmath}
\usepackage{epic}
\usepackage{epsfig}
\usepackage{fontenc}
\usepackage{framed,color}
\usepackage{palatino, url, multicol}
\usepackage{listings}
%\algsetup{indent=2em}


\vspace{-0.5cm}
\title{Drawing Estimates from the Posterior}
\vspace{-0.5cm}
\author[Felipe Bravo Márquez]{\footnotesize
%\author{\footnotesize  
 \textcolor[rgb]{0.00,0.00,1.00}{Felipe José Bravo Márquez}} 
\date{ \today }




\begin{document}
\begin{frame}
\titlepage


\end{frame}


%%%%%%%%%%%%%%%%%%%%%%%%%%%


\begin{frame}{Drawing Estimates from the Posterior}
\scriptsize{
\begin{itemize}

\item In this class we will learn how to draw estimates (e.g., point estimates, intervals) from a posterior distribution.

\item Once your Bayesian model produces a posterior distribution, the model's work is done.

\item But your work has just begun. 

\item It is necessary to summarize and interpret the posterior distribution.

\item Exactly how it is summarized depends upon your purpose.

\item Common questions include:
\begin{itemize}
\begin{scriptsize}
 \item How much posterior probability lies below some parameter value?
 \item How much posterior probability lies between two parameter values?
 \item Which parameter value marks the lower 5\% of the posterior probability?
 \item Which range of parameter values contains 90\% of the posterior probability?
 \item Which parameter value has highest posterior probability?
 \end{scriptsize}
\end{itemize}
 
\end{itemize}



} 

\end{frame}


\begin{frame}{Sampling to summarize}
\scriptsize{
\begin{itemize}

\item These simple questions can be usefully divided into questions about: 
\begin{itemize}
\scriptsize{
 \item intervals of defined boundaries
 \item intervals of defined probability mass
 \item point estimates
}
\end{itemize}
 
\item In the theoretical world (when the posterior has a closed mathematical expressions), answering these questions implies calculating complicated integrals to cancel out (or average) different variables.

\item In the practical world, however, the same results can be approximated  using \textbf{samples} from the posterior. 
 
\item In this class we will approach the above questions using samples from the posterior. 

\item Another reason to learn to work with posterior samples is that methods like MCMC produce nothing but samples from the posterior.


\item This class is based on Chapter 3 of \cite{mcelreath2020statistical}.

\end{itemize}
}

\end{frame}










\begin{frame}{Sampling from a grid-approximate posterior}
\scriptsize{
\begin{itemize}

\item Before beginning to work with samples, we need to generate them.

\item Here’s a reminder for how to compute the posterior for the globe tossing model, using grid approximation:

 
\end{itemize}



} 

\end{frame}


\begin{frame}{Conclusions}
\scriptsize{

\begin{itemize}
\item Blabla
\end{itemize}


} 
\end{frame}


%%%%%%%%%%%%%%%%%%%%%%%%%%%
\begin{frame}[allowframebreaks]\scriptsize
\frametitle{References}
\bibliography{bio}
\bibliographystyle{apalike}
%\bibliographystyle{flexbib}
\end{frame}  









%%%%%%%%%%%%%%%%%%%%%%%%%%%

\end{document}
